% Options for packages loaded elsewhere
\PassOptionsToPackage{unicode}{hyperref}
\PassOptionsToPackage{hyphens}{url}
%
\documentclass[
  12,
]{article}
\usepackage{amsmath,amssymb}
\usepackage{lmodern}
\usepackage{setspace}
\usepackage{iftex}
\ifPDFTeX
  \usepackage[T1]{fontenc}
  \usepackage[utf8]{inputenc}
  \usepackage{textcomp} % provide euro and other symbols
\else % if luatex or xetex
  \usepackage{unicode-math}
  \defaultfontfeatures{Scale=MatchLowercase}
  \defaultfontfeatures[\rmfamily]{Ligatures=TeX,Scale=1}
\fi
% Use upquote if available, for straight quotes in verbatim environments
\IfFileExists{upquote.sty}{\usepackage{upquote}}{}
\IfFileExists{microtype.sty}{% use microtype if available
  \usepackage[]{microtype}
  \UseMicrotypeSet[protrusion]{basicmath} % disable protrusion for tt fonts
}{}
\makeatletter
\@ifundefined{KOMAClassName}{% if non-KOMA class
  \IfFileExists{parskip.sty}{%
    \usepackage{parskip}
  }{% else
    \setlength{\parindent}{0pt}
    \setlength{\parskip}{6pt plus 2pt minus 1pt}}
}{% if KOMA class
  \KOMAoptions{parskip=half}}
\makeatother
\usepackage{xcolor}
\IfFileExists{xurl.sty}{\usepackage{xurl}}{} % add URL line breaks if available
\IfFileExists{bookmark.sty}{\usepackage{bookmark}}{\usepackage{hyperref}}
\hypersetup{
  pdftitle={fasteR},
  pdfauthor={Hyeongchan Bae},
  hidelinks,
  pdfcreator={LaTeX via pandoc}}
\urlstyle{same} % disable monospaced font for URLs
\usepackage[margin=1in]{geometry}
\usepackage{color}
\usepackage{fancyvrb}
\newcommand{\VerbBar}{|}
\newcommand{\VERB}{\Verb[commandchars=\\\{\}]}
\DefineVerbatimEnvironment{Highlighting}{Verbatim}{commandchars=\\\{\}}
% Add ',fontsize=\small' for more characters per line
\usepackage{framed}
\definecolor{shadecolor}{RGB}{248,248,248}
\newenvironment{Shaded}{\begin{snugshade}}{\end{snugshade}}
\newcommand{\AlertTok}[1]{\textcolor[rgb]{0.94,0.16,0.16}{#1}}
\newcommand{\AnnotationTok}[1]{\textcolor[rgb]{0.56,0.35,0.01}{\textbf{\textit{#1}}}}
\newcommand{\AttributeTok}[1]{\textcolor[rgb]{0.77,0.63,0.00}{#1}}
\newcommand{\BaseNTok}[1]{\textcolor[rgb]{0.00,0.00,0.81}{#1}}
\newcommand{\BuiltInTok}[1]{#1}
\newcommand{\CharTok}[1]{\textcolor[rgb]{0.31,0.60,0.02}{#1}}
\newcommand{\CommentTok}[1]{\textcolor[rgb]{0.56,0.35,0.01}{\textit{#1}}}
\newcommand{\CommentVarTok}[1]{\textcolor[rgb]{0.56,0.35,0.01}{\textbf{\textit{#1}}}}
\newcommand{\ConstantTok}[1]{\textcolor[rgb]{0.00,0.00,0.00}{#1}}
\newcommand{\ControlFlowTok}[1]{\textcolor[rgb]{0.13,0.29,0.53}{\textbf{#1}}}
\newcommand{\DataTypeTok}[1]{\textcolor[rgb]{0.13,0.29,0.53}{#1}}
\newcommand{\DecValTok}[1]{\textcolor[rgb]{0.00,0.00,0.81}{#1}}
\newcommand{\DocumentationTok}[1]{\textcolor[rgb]{0.56,0.35,0.01}{\textbf{\textit{#1}}}}
\newcommand{\ErrorTok}[1]{\textcolor[rgb]{0.64,0.00,0.00}{\textbf{#1}}}
\newcommand{\ExtensionTok}[1]{#1}
\newcommand{\FloatTok}[1]{\textcolor[rgb]{0.00,0.00,0.81}{#1}}
\newcommand{\FunctionTok}[1]{\textcolor[rgb]{0.00,0.00,0.00}{#1}}
\newcommand{\ImportTok}[1]{#1}
\newcommand{\InformationTok}[1]{\textcolor[rgb]{0.56,0.35,0.01}{\textbf{\textit{#1}}}}
\newcommand{\KeywordTok}[1]{\textcolor[rgb]{0.13,0.29,0.53}{\textbf{#1}}}
\newcommand{\NormalTok}[1]{#1}
\newcommand{\OperatorTok}[1]{\textcolor[rgb]{0.81,0.36,0.00}{\textbf{#1}}}
\newcommand{\OtherTok}[1]{\textcolor[rgb]{0.56,0.35,0.01}{#1}}
\newcommand{\PreprocessorTok}[1]{\textcolor[rgb]{0.56,0.35,0.01}{\textit{#1}}}
\newcommand{\RegionMarkerTok}[1]{#1}
\newcommand{\SpecialCharTok}[1]{\textcolor[rgb]{0.00,0.00,0.00}{#1}}
\newcommand{\SpecialStringTok}[1]{\textcolor[rgb]{0.31,0.60,0.02}{#1}}
\newcommand{\StringTok}[1]{\textcolor[rgb]{0.31,0.60,0.02}{#1}}
\newcommand{\VariableTok}[1]{\textcolor[rgb]{0.00,0.00,0.00}{#1}}
\newcommand{\VerbatimStringTok}[1]{\textcolor[rgb]{0.31,0.60,0.02}{#1}}
\newcommand{\WarningTok}[1]{\textcolor[rgb]{0.56,0.35,0.01}{\textbf{\textit{#1}}}}
\usepackage{graphicx}
\makeatletter
\def\maxwidth{\ifdim\Gin@nat@width>\linewidth\linewidth\else\Gin@nat@width\fi}
\def\maxheight{\ifdim\Gin@nat@height>\textheight\textheight\else\Gin@nat@height\fi}
\makeatother
% Scale images if necessary, so that they will not overflow the page
% margins by default, and it is still possible to overwrite the defaults
% using explicit options in \includegraphics[width, height, ...]{}
\setkeys{Gin}{width=\maxwidth,height=\maxheight,keepaspectratio}
% Set default figure placement to htbp
\makeatletter
\def\fps@figure{htbp}
\makeatother
\setlength{\emergencystretch}{3em} % prevent overfull lines
\providecommand{\tightlist}{%
  \setlength{\itemsep}{0pt}\setlength{\parskip}{0pt}}
\setcounter{secnumdepth}{-\maxdimen} % remove section numbering
\ifLuaTeX
  \usepackage{selnolig}  % disable illegal ligatures
\fi

\title{fasteR}
\author{Hyeongchan Bae}
\date{April 2022}

\begin{document}
\maketitle

{
\setcounter{tocdepth}{2}
\tableofcontents
}
\setstretch{1.5}
\hypertarget{day-1}{%
\section{Day 1}\label{day-1}}

\textbf{R과 통계분석 (Tidyverse 활용) p.4\textasciitilde60}

\hypertarget{download}{%
\subsection{1. Download}\label{download}}

\textbf{R} 프로그래밍 언어. 제일 먼저 설치

\begin{itemize}
\tightlist
\item
  \url{https://cran.r-project.org/bin/windows/base}
\end{itemize}

\textbf{RStudio} R을 활용하기 위한 통합개발환경(IDE, Integrated
Development Environment)

\begin{itemize}
\item
  달리 말하면 RStudio 외에도 다양한 프로그램에서 R을 사용할 수 있음.
  여러 언어를 사용하는 개발자는 VS Code 같은 단일 IDE를 활용하기도.
\item
  \url{https://www.rstudio.com/products/rstudio/download}
\end{itemize}

\textbf{Rtools} 패키지를 설치하다 보면 필요한 경우(Compile)가 더러 있음

\begin{itemize}
\tightlist
\item
  \url{https://cran.r-project.org/bin/windows/Rtools}
\end{itemize}

\textbf{Chrome Browser} Selenium 파트에서 사용할 예정

\begin{itemize}
\tightlist
\item
  \url{https://www.google.com/intl/ko_kr/chrome}
\end{itemize}

\textbf{Extra Files} 인턴의 깃허브 페이지. 강의 노트, 작성 코드 등
추가적인 파일을 업로드 해둠

\begin{itemize}
\tightlist
\item
  \url{https://github.com/Rlearnchan/fasteR}
\end{itemize}

\hypertarget{something-to-know}{%
\subsection{2. Something to know}\label{something-to-know}}

\hypertarget{uxbaa8uxb450-uxb2e4-uxac19uxc740-uxc544uxb9c8uxcd94uxc5b4uxc57c}{%
\subsubsection{1) ``모두 다 같은
아마추어야''}\label{uxbaa8uxb450-uxb2e4-uxac19uxc740-uxc544uxb9c8uxcd94uxc5b4uxc57c}}

\begin{itemize}
\item
  익숙지 않고, 오래 걸리는 게 당연해요.
\item
  모든 과정을 R로 수행할 필요는 없으니, 잘 안 되면 데이터를 엑셀, STATA
  등으로 옮겨 처리해오셔도 좋습니다.
\end{itemize}

\hypertarget{r-uxd30cuxc77cuxd655uxc7a5uxc790}{%
\subsubsection{2) R
파일(확장자)}\label{r-uxd30cuxc77cuxd655uxc7a5uxc790}}

\textbf{.R} 작성한 코드

\begin{itemize}
\item
  좌상단 Script에서 코드 작성 \(\to\) 필요한 부분 실행 \(\to\) 좌하단
  Console 창에서 실행 결과를 확인하는 게 일반적이라, 작업 첫 파트를
  저장한 것이라 볼 수 있음.
\item
  코드를 적은 메모장쯤 되니, \textbf{비슷한 환경}이라면 타인에게 받은
  코드를 실행하기만 해도 같은 결과를 시현함.
\end{itemize}

\textbf{.Rdata} 작업공간 이미지

\begin{itemize}
\item
  우상단 Environment에 기록된, 작업하며 생성된 객체, 함수, 기타
  데이터들의 총체.
\item
  임시로 만든, 코드 실행과 무관환 object 들도 저장.
\end{itemize}

\textbf{.Rhistory} 작업 기록

\begin{itemize}
\tightlist
\item
  RStudio를 종료했다가, 다시 실행하면 이전 작업 상태가 비교적 온전히
  남아있는데, 이를 위한 파일이라 하겠음.
\end{itemize}

\textbf{.Rmd} 마크다운 파일

\begin{itemize}
\item
  html, pdf, word 등을 만들기 위해 Markdown 문법으로 작성한, 코드 친구쯤
  되는 녀석.
\item
  본 문서도 마크다운으로 작성. 소위 `R로 논문 쓴다' 할 때 등장.
\end{itemize}

\hypertarget{uxd55cuxae00uxc5d0-uxc720uxb3c5-uxcde8uxc57duxd55c-r}{%
\subsubsection{3) 한글에 유독 취약한
R}\label{uxd55cuxae00uxc5d0-uxc720uxb3c5-uxcde8uxc57duxd55c-r}}

\textbf{UTF-8} 인코딩 방식을 변경해주세요.

\begin{itemize}
\item
  Tools \(\to\) Global Options \(\to\) Code \(\to\) Saving 경로.
\item
  타인에게 받은, 혹은 건네준 코드 파일에서 한글이 깨져 보인다면 대체로
  이 문제.
\item
  File \(\to\) Reopen with Encoding 기능을 활용해 대처하는 방법도 있음.
\end{itemize}

\textbf{Library Path} 패키지 설치 경로에 한글 네이밍이 없도록 해주세요.

\begin{Shaded}
\begin{Highlighting}[]
\FunctionTok{.libPaths}\NormalTok{() }\CommentTok{\# 첫 번째가 default. 개인 폴더가 설정돼 두 개 나오기도 한다.}
\end{Highlighting}
\end{Shaded}

\begin{verbatim}
## [1] "C:/Program Files/R/R-4.1.3/library"
\end{verbatim}

\begin{itemize}
\item
  (윈도우 기준 예시) 만일 {[}2{]} ``C:/Users/\textbf{사용자
  이름}/Documents/R/win-library/4.1.3'' 경로에 한글이 포함된다면,
  패키지를 다룰 때마다 오류 사인을 접할 공산이 큼.
\item
  새로 계정을 만들지 않는 이상, \textbf{사용자 이름} 구간 폴더 명은
  변경하기도 어려움.
\item
  다음과 같이 \textbf{사용자 이름}이 없는 기본(공용) 라이브러리를
  default로 설정하길 권장.
\end{itemize}

\begin{Shaded}
\begin{Highlighting}[]
\FunctionTok{Sys.setenv}\NormalTok{(}\StringTok{\textquotesingle{}R\_LIBS\_USER\textquotesingle{}} \OtherTok{=} \StringTok{\textquotesingle{}C:/Program Files/R/R{-}4.1.3/library\textquotesingle{}}\NormalTok{) }
\CommentTok{\# R의 \textquotesingle{}개인\textquotesingle{} 세팅을 앞서 발견한 \textquotesingle{}기본(공용)\textquotesingle{} 경로로 덮어쓰기.}

\FunctionTok{.libPaths}\NormalTok{(}\StringTok{\textquotesingle{}R\_LIBS\_USER\textquotesingle{}}\NormalTok{) }\CommentTok{\# 바뀐 \textquotesingle{}개인 라이브러리\textquotesingle{}를 패키지 설치 경로로 설정.}
\end{Highlighting}
\end{Shaded}

\begin{Shaded}
\begin{Highlighting}[]
\FunctionTok{.libPaths}\NormalTok{() }\CommentTok{\# 하나의 경로로 잘 세팅되고,}
\end{Highlighting}
\end{Shaded}

\begin{verbatim}
## [1] "C:/Program Files/R/R-4.1.3/library"
\end{verbatim}

\begin{Shaded}
\begin{Highlighting}[]
\FunctionTok{.libPaths}\NormalTok{() }\SpecialCharTok{==} \FunctionTok{Sys.getenv}\NormalTok{(}\StringTok{\textquotesingle{}R\_LIBS\_USER\textquotesingle{}}\NormalTok{) }\CommentTok{\# 개인 라이브러리 경로와도 일치}
\end{Highlighting}
\end{Shaded}

\begin{verbatim}
## [1] TRUE
\end{verbatim}

\hypertarget{uxc0acuxc6a9uxbc95}{%
\subsubsection{4) ?, ?? 사용법}\label{uxc0acuxc6a9uxbc95}}

\textbf{?} 모르는 함수 검색하기

\begin{Shaded}
\begin{Highlighting}[]
\NormalTok{?print}
\end{Highlighting}
\end{Shaded}

\begin{itemize}
\item
  대부분의 함수는 R Documentation 이라 해서 정의와 기능, 인자, 간단한
  사용 예시 등을 요약해둔 페이지를 가지고 있음.
\item
  예컨대 \texttt{print()} 함수를 자세히 알고 싶다면, 위에서 처럼
  \texttt{?} 하나 붙여서 실행하면 됨.
\end{itemize}

\textbf{??} 모르는 개념, 워딩 검색하기

\begin{Shaded}
\begin{Highlighting}[]
\NormalTok{??print}
\end{Highlighting}
\end{Shaded}

\begin{itemize}
\item
  하지만 함수 이름조차 모르거나, 기능을 연상할 키워드 정도만 간신히 아는
  경우도 많음.
\item
  \texttt{??}는 모든 R Documentation 에서 해당 단어가 포함된 것을 모두
  골라 보여줌.
\item
  두 가지를 적절히 섞어 사용하는 게 좋음.
\end{itemize}

\hypertarget{uxad6cuxae00uxb9c1}{%
\subsubsection{5) 구글링}\label{uxad6cuxae00uxb9c1}}

\begin{itemize}
\item
  사실 구글은 모든 걸 알고 있음.
\item
  \texttt{str\_dectect()} 식으로 함수 이름 자체를 검색하면 국내외
  사용자들이 포스팅한 글을 찾아보기 편함.
\item
  \texttt{warning} 혹은 \texttt{error} 사인은 해당 문구를 적당히 복사해
  구글에 그대로 쳐보는 게 좋음.
\item
  stackoverflow 같은 개발자 커뮤니티 게시물이 주로 나올 텐데, 같은
  문제로 고민한 사람들이 꽤 많았기 때문.
\end{itemize}

\hypertarget{basic-function}{%
\subsection{3. Basic Function}\label{basic-function}}

\hypertarget{uxc22buxc790-uxacc4uxc0b0}{%
\subsubsection{1) 숫자 계산}\label{uxc22buxc790-uxacc4uxc0b0}}

\begin{Shaded}
\begin{Highlighting}[]
\DecValTok{3}\SpecialCharTok{+}\DecValTok{4{-}7}\SpecialCharTok{/}\DecValTok{3} \CommentTok{\# 달리 명령어가 필요하진 않으나,}
\end{Highlighting}
\end{Shaded}

\begin{verbatim}
## [1] 4.666667
\end{verbatim}

\begin{Shaded}
\begin{Highlighting}[]
\FunctionTok{print}\NormalTok{(}\DecValTok{3}\SpecialCharTok{+}\DecValTok{4{-}7}\SpecialCharTok{/}\DecValTok{3}\NormalTok{) }\CommentTok{\# print() 함수를 사용할 수도 있음}
\end{Highlighting}
\end{Shaded}

\begin{verbatim}
## [1] 4.666667
\end{verbatim}

\begin{Shaded}
\begin{Highlighting}[]
\FunctionTok{print}\NormalTok{(}\DecValTok{3}\SpecialCharTok{+}\DecValTok{4{-}7}\SpecialCharTok{/}\DecValTok{3}\NormalTok{, }\AttributeTok{digits =} \DecValTok{3}\NormalTok{) }\CommentTok{\# 세 번째 자리에서 반올림}
\end{Highlighting}
\end{Shaded}

\begin{verbatim}
## [1] 4.67
\end{verbatim}

\begin{Shaded}
\begin{Highlighting}[]
\FunctionTok{rnorm}\NormalTok{(}\AttributeTok{n =} \DecValTok{5}\NormalTok{, }\AttributeTok{mean =} \DecValTok{0}\NormalTok{, }\AttributeTok{sd =} \DecValTok{1}\NormalTok{) }\CommentTok{\# n(0, 1) 분포에서 5개 난수 생성}
\end{Highlighting}
\end{Shaded}

\begin{verbatim}
## [1]  0.7764636 -0.4945202  0.4896974 -0.5007566 -0.1512130
\end{verbatim}

\begin{Shaded}
\begin{Highlighting}[]
\NormalTok{stats}\SpecialCharTok{::}\FunctionTok{rnorm}\NormalTok{(}\AttributeTok{n =} \DecValTok{5}\NormalTok{, }\AttributeTok{mean =} \DecValTok{0}\NormalTok{, }\AttributeTok{sd =} \DecValTok{1}\NormalTok{) }\CommentTok{\# stats 패키지의 rnorm() 함수}
\end{Highlighting}
\end{Shaded}

\begin{verbatim}
## [1] -0.5528755  0.2924098 -0.5191754 -0.8114824  0.3399659
\end{verbatim}

\begin{itemize}
\item
  기본 패키지, 혹은 \texttt{library()} 로 장착한 패키지의 함수는
  \texttt{::} 표기를 사용하지 않아도 됨.
\item
  여러 패키지를 동시에 사용하는 경우, 유사한 이름의 함수 간 혼동을
  피하기 위해 \texttt{::} 방식을 사용하기도 함.
\end{itemize}

\begin{Shaded}
\begin{Highlighting}[]
\FunctionTok{set.seed}\NormalTok{(}\AttributeTok{seed =} \DecValTok{10}\NormalTok{)}
\end{Highlighting}
\end{Shaded}

\begin{itemize}
\tightlist
\item
  reproducibility 위해서 난수 생성 규칙을 \texttt{set.seed()}로 부여.
\end{itemize}

\begin{Shaded}
\begin{Highlighting}[]
\FunctionTok{rnorm}\NormalTok{(}\DecValTok{5}\NormalTok{, }\DecValTok{0}\NormalTok{, }\DecValTok{1}\NormalTok{)}
\end{Highlighting}
\end{Shaded}

\begin{verbatim}
## [1]  0.01874617 -0.18425254 -1.37133055 -0.59916772  0.29454513
\end{verbatim}

\hypertarget{uxd14duxc2a4uxd2b8}{%
\subsubsection{2) 텍스트}\label{uxd14duxc2a4uxd2b8}}

\begin{Shaded}
\begin{Highlighting}[]
\StringTok{\textquotesingle{}banana\textquotesingle{}} \CommentTok{\# 작은 따옴표}
\end{Highlighting}
\end{Shaded}

\begin{verbatim}
## [1] "banana"
\end{verbatim}

\begin{Shaded}
\begin{Highlighting}[]
\StringTok{"banana"} \CommentTok{\# 큰 따옴표 모두 사용 가능}
\end{Highlighting}
\end{Shaded}

\begin{verbatim}
## [1] "banana"
\end{verbatim}

\begin{Shaded}
\begin{Highlighting}[]
\FunctionTok{class}\NormalTok{(}\StringTok{\textquotesingle{}banana\textquotesingle{}}\NormalTok{) }\CommentTok{\# 문자 클래스}
\end{Highlighting}
\end{Shaded}

\begin{verbatim}
## [1] "character"
\end{verbatim}

\begin{itemize}
\tightlist
\item
  \texttt{class()} 함수는 자주 쓰니 기억해 둘 필요가 있음.
\end{itemize}

\begin{Shaded}
\begin{Highlighting}[]
\FunctionTok{paste0}\NormalTok{(}\StringTok{\textquotesingle{}이제와\textquotesingle{}}\NormalTok{, }\StringTok{\textquotesingle{}뒤늦게\textquotesingle{}}\NormalTok{, }\StringTok{\textquotesingle{}무엇을 더 보태려하나\textquotesingle{}}\NormalTok{) }\CommentTok{\# 문자열 붙여서 하나로 만듦}
\end{Highlighting}
\end{Shaded}

\begin{verbatim}
## [1] "이제와뒤늦게무엇을 더 보태려하나"
\end{verbatim}

\begin{Shaded}
\begin{Highlighting}[]
\FunctionTok{paste0}\NormalTok{(}\StringTok{\textquotesingle{}이제와 \textquotesingle{}}\NormalTok{, }\StringTok{\textquotesingle{}뒤늦게 \textquotesingle{}}\NormalTok{, }\StringTok{\textquotesingle{}무엇을 더 보태려하나\textquotesingle{}}\NormalTok{) }\CommentTok{\# 띄어쓰기를 포함해서 붙이는 게 요령}
\end{Highlighting}
\end{Shaded}

\begin{verbatim}
## [1] "이제와 뒤늦게 무엇을 더 보태려하나"
\end{verbatim}

\begin{Shaded}
\begin{Highlighting}[]
\FunctionTok{paste}\NormalTok{(}\StringTok{\textquotesingle{}이제와\textquotesingle{}}\NormalTok{, }\StringTok{\textquotesingle{}뒤늦게\textquotesingle{}}\NormalTok{, }\StringTok{\textquotesingle{}무엇을 더 보태려하나\textquotesingle{}}\NormalTok{) }\CommentTok{\# 한 칸씩 띄어쓰는 게 default인 함수}
\end{Highlighting}
\end{Shaded}

\begin{verbatim}
## [1] "이제와 뒤늦게 무엇을 더 보태려하나"
\end{verbatim}

\begin{Shaded}
\begin{Highlighting}[]
\FunctionTok{paste}\NormalTok{(}\StringTok{\textquotesingle{}이제와\textquotesingle{}}\NormalTok{, }\StringTok{\textquotesingle{}뒤늦게\textquotesingle{}}\NormalTok{, }\StringTok{\textquotesingle{}무엇을 더 보태려하나\textquotesingle{}}\NormalTok{, }\AttributeTok{sep =} \StringTok{\textquotesingle{}둠칫\textquotesingle{}}\NormalTok{) }\CommentTok{\# 사실 sep = \textquotesingle{} \textquotesingle{} 인자가 숨어있던 것. 바꿀 수도 있음.}
\end{Highlighting}
\end{Shaded}

\begin{verbatim}
## [1] "이제와둠칫뒤늦게둠칫무엇을 더 보태려하나"
\end{verbatim}

\hypertarget{uxac1duxccb4}{%
\subsubsection{3) 객체}\label{uxac1duxccb4}}

\begin{Shaded}
\begin{Highlighting}[]
\NormalTok{BR31 }\OtherTok{=} \StringTok{\textquotesingle{}Alien Mom\textquotesingle{}} \CommentTok{\# 텍스트를 BR31 객체에 저장}

\NormalTok{br31 }\OtherTok{=} \StringTok{\textquotesingle{}Mint Choco\textquotesingle{}} \CommentTok{\# 텍스트를 br31 객체에 저장}
\end{Highlighting}
\end{Shaded}

\begin{Shaded}
\begin{Highlighting}[]
\NormalTok{BR31}
\end{Highlighting}
\end{Shaded}

\begin{verbatim}
## [1] "Alien Mom"
\end{verbatim}

\begin{Shaded}
\begin{Highlighting}[]
\NormalTok{br31}
\end{Highlighting}
\end{Shaded}

\begin{verbatim}
## [1] "Mint Choco"
\end{verbatim}

\begin{Shaded}
\begin{Highlighting}[]
\FunctionTok{paste}\NormalTok{(}\StringTok{\textquotesingle{}Which do you prefer\textquotesingle{}}\NormalTok{, BR31, }\StringTok{\textquotesingle{}or\textquotesingle{}}\NormalTok{, br31) }\CommentTok{\# 객체명을 입력하면 담긴 것을 가져다 씀.}
\end{Highlighting}
\end{Shaded}

\begin{verbatim}
## [1] "Which do you prefer Alien Mom or Mint Choco"
\end{verbatim}

\begin{itemize}
\tightlist
\item
  객체 명을 지을 땐 \textbf{대소문자 구별}, 그리고 \textbf{첫 글자엔
  숫자 및 기호 불가} 특성을 고려해야 함.
\end{itemize}

\hypertarget{uxbca1uxd130}{%
\subsubsection{4) 벡터}\label{uxbca1uxd130}}

\begin{Shaded}
\begin{Highlighting}[]
\NormalTok{Yunha }\OtherTok{=} \FunctionTok{c}\NormalTok{(}\DecValTok{4}\NormalTok{, }\DecValTok{8}\NormalTok{, }\DecValTok{6}\NormalTok{) }\CommentTok{\# 숫자 세 개를 벡터로 묶어 저장}
\end{Highlighting}
\end{Shaded}

\begin{Shaded}
\begin{Highlighting}[]
\NormalTok{Yunha}
\end{Highlighting}
\end{Shaded}

\begin{verbatim}
## [1] 4 8 6
\end{verbatim}

\begin{Shaded}
\begin{Highlighting}[]
\FunctionTok{class}\NormalTok{(Yunha) }\CommentTok{\# 숫자 속성이 그대로 남아 있음}
\end{Highlighting}
\end{Shaded}

\begin{verbatim}
## [1] "numeric"
\end{verbatim}

\begin{Shaded}
\begin{Highlighting}[]
\NormalTok{Yunha }\OtherTok{=} \FunctionTok{c}\NormalTok{(}\StringTok{\textquotesingle{}Password\textquotesingle{}}\NormalTok{, }\DecValTok{4}\NormalTok{, }\DecValTok{8}\NormalTok{, }\DecValTok{6}\NormalTok{) }\CommentTok{\# 원소가 하나라도 character가 섞이면}
\end{Highlighting}
\end{Shaded}

\begin{Shaded}
\begin{Highlighting}[]
\NormalTok{Yunha }\CommentTok{\# 따옴표 찍힌 것부터 느낌이 다르고,}
\end{Highlighting}
\end{Shaded}

\begin{verbatim}
## [1] "Password" "4"        "8"        "6"
\end{verbatim}

\begin{Shaded}
\begin{Highlighting}[]
\FunctionTok{class}\NormalTok{(Yunha) }\CommentTok{\# 얄짤없이 전부 character로 저장}
\end{Highlighting}
\end{Shaded}

\begin{verbatim}
## [1] "character"
\end{verbatim}

\hypertarget{uxd589uxb82c}{%
\subsubsection{5) 행렬}\label{uxd589uxb82c}}

\begin{Shaded}
\begin{Highlighting}[]
\FunctionTok{matrix}\NormalTok{(}\AttributeTok{data =} \FunctionTok{c}\NormalTok{(}\DecValTok{1}\NormalTok{, }\DecValTok{2}\NormalTok{, }\DecValTok{3}\NormalTok{, }\DecValTok{4}\NormalTok{, }\DecValTok{5}\NormalTok{, }\DecValTok{6}\NormalTok{, }\DecValTok{7}\NormalTok{, }\DecValTok{8}\NormalTok{, }\DecValTok{9}\NormalTok{, }\DecValTok{10}\NormalTok{, }\DecValTok{11}\NormalTok{, }\DecValTok{12}\NormalTok{), }\AttributeTok{nrow =} \DecValTok{3}\NormalTok{, }\AttributeTok{ncol =} \DecValTok{4}\NormalTok{)}
\end{Highlighting}
\end{Shaded}

\begin{verbatim}
##      [,1] [,2] [,3] [,4]
## [1,]    1    4    7   10
## [2,]    2    5    8   11
## [3,]    3    6    9   12
\end{verbatim}

\begin{Shaded}
\begin{Highlighting}[]
\FunctionTok{matrix}\NormalTok{(}\AttributeTok{data =} \FunctionTok{c}\NormalTok{(}\DecValTok{1}\NormalTok{, }\DecValTok{2}\NormalTok{, }\DecValTok{3}\NormalTok{, }\DecValTok{4}\NormalTok{, }\DecValTok{5}\NormalTok{, }\DecValTok{6}\NormalTok{, }\DecValTok{7}\NormalTok{, }\DecValTok{8}\NormalTok{, }\DecValTok{9}\NormalTok{, }\DecValTok{10}\NormalTok{, }\DecValTok{11}\NormalTok{, }\DecValTok{12}\NormalTok{), }\AttributeTok{nrow =} \DecValTok{3}\NormalTok{, }\AttributeTok{ncol =} \DecValTok{4}\NormalTok{, }\AttributeTok{dimnames =} \FunctionTok{list}\NormalTok{(}\FunctionTok{c}\NormalTok{(}\StringTok{\textquotesingle{}가\textquotesingle{}}\NormalTok{, }\StringTok{\textquotesingle{}나\textquotesingle{}}\NormalTok{, }\StringTok{\textquotesingle{}다\textquotesingle{}}\NormalTok{), }\FunctionTok{c}\NormalTok{(}\StringTok{\textquotesingle{}A\textquotesingle{}}\NormalTok{, }\StringTok{\textquotesingle{}B\textquotesingle{}}\NormalTok{, }\StringTok{\textquotesingle{}C\textquotesingle{}}\NormalTok{, }\StringTok{\textquotesingle{}D\textquotesingle{}}\NormalTok{))) }\CommentTok{\# 행이름, 열이름 설정}
\end{Highlighting}
\end{Shaded}

\begin{verbatim}
##    A B C  D
## 가 1 4 7 10
## 나 2 5 8 11
## 다 3 6 9 12
\end{verbatim}

\begin{Shaded}
\begin{Highlighting}[]
\FunctionTok{matrix}\NormalTok{(}\AttributeTok{data =} \FunctionTok{c}\NormalTok{(}\DecValTok{1}\NormalTok{, }\DecValTok{2}\NormalTok{, }\DecValTok{3}\NormalTok{, }\DecValTok{4}\NormalTok{, }\DecValTok{5}\NormalTok{, }\DecValTok{6}\NormalTok{, }\DecValTok{7}\NormalTok{, }\DecValTok{8}\NormalTok{, }\DecValTok{9}\NormalTok{, }\DecValTok{10}\NormalTok{, }\DecValTok{11}\NormalTok{, }\DecValTok{12}\NormalTok{),}
       \AttributeTok{dimnames =} \FunctionTok{list}\NormalTok{(}\FunctionTok{c}\NormalTok{(}\StringTok{\textquotesingle{}가\textquotesingle{}}\NormalTok{, }\StringTok{\textquotesingle{}나\textquotesingle{}}\NormalTok{, }\StringTok{\textquotesingle{}다\textquotesingle{}}\NormalTok{), }\FunctionTok{c}\NormalTok{(}\StringTok{\textquotesingle{}A\textquotesingle{}}\NormalTok{, }\StringTok{\textquotesingle{}B\textquotesingle{}}\NormalTok{, }\StringTok{\textquotesingle{}C\textquotesingle{}}\NormalTok{, }\StringTok{\textquotesingle{}D\textquotesingle{}}\NormalTok{)),}
       \AttributeTok{nrow =} \DecValTok{3}\NormalTok{, }\AttributeTok{ncol =} \DecValTok{4}\NormalTok{) }\CommentTok{\# 코드가 슬슬 길어지니, 엔터를 적극 활용.}
\end{Highlighting}
\end{Shaded}

\begin{verbatim}
##    A B C  D
## 가 1 4 7 10
## 나 2 5 8 11
## 다 3 6 9 12
\end{verbatim}

\begin{itemize}
\item
  마지막 코드에선 인자들의 순서가 조금 다른데, 이는 인자명을 지정해줬기
  때문에 가능한 것.
\item
  지정만 잘 돼있으면 섞여도 상관 없음.
\end{itemize}

\begin{Shaded}
\begin{Highlighting}[]
\NormalTok{mat }\OtherTok{\textless{}{-}} \FunctionTok{matrix}\NormalTok{(}\DecValTok{1}\SpecialCharTok{:}\DecValTok{12}\NormalTok{, }\DecValTok{3}\NormalTok{, }\DecValTok{4}\NormalTok{) }\CommentTok{\# matrix() 함수의 처음 세 인자가 data, nrow, ncol 이므로 필요한 값만 입력.}

\FunctionTok{colnames}\NormalTok{(mat) }\OtherTok{\textless{}{-}} \FunctionTok{c}\NormalTok{(}\StringTok{\textquotesingle{}A\textquotesingle{}}\NormalTok{, }\StringTok{\textquotesingle{}B\textquotesingle{}}\NormalTok{, }\StringTok{\textquotesingle{}C\textquotesingle{}}\NormalTok{, }\StringTok{\textquotesingle{}D\textquotesingle{}}\NormalTok{) }\CommentTok{\# 열이름 덮어쓰기}

\FunctionTok{rownames}\NormalTok{(mat) }\OtherTok{\textless{}{-}} \FunctionTok{c}\NormalTok{(}\StringTok{\textquotesingle{}가\textquotesingle{}}\NormalTok{, }\StringTok{\textquotesingle{}나\textquotesingle{}}\NormalTok{, }\StringTok{\textquotesingle{}다\textquotesingle{}}\NormalTok{) }\CommentTok{\# 행이름 덮어쓰기}
\end{Highlighting}
\end{Shaded}

\begin{itemize}
\item
  하다 보면 쉬운 길을 찾아가게 됨.
\item
  \texttt{1:12}는 \texttt{seq(from\ =\ 1,\ to\ =\ 12,\ by\ =\ 1)} 과
  같음.
\item
  \texttt{\textless{}-}는 \texttt{=}과 같음.
\end{itemize}

\begin{Shaded}
\begin{Highlighting}[]
\NormalTok{mat}
\end{Highlighting}
\end{Shaded}

\begin{verbatim}
##    A B C  D
## 가 1 4 7 10
## 나 2 5 8 11
## 다 3 6 9 12
\end{verbatim}

\begin{Shaded}
\begin{Highlighting}[]
\FunctionTok{class}\NormalTok{(mat)}
\end{Highlighting}
\end{Shaded}

\begin{verbatim}
## [1] "matrix" "array"
\end{verbatim}

\hypertarget{uxb370uxc774uxd130uxd504uxb808uxc784}{%
\subsubsection{6)
데이터프레임}\label{uxb370uxc774uxd130uxd504uxb808uxc784}}

\begin{Shaded}
\begin{Highlighting}[]
\NormalTok{윤하 }\OtherTok{\textless{}{-}} \FunctionTok{as.data.frame}\NormalTok{(Yunha)}

\NormalTok{matthew }\OtherTok{\textless{}{-}} \FunctionTok{as.data.frame}\NormalTok{(mat)}
\end{Highlighting}
\end{Shaded}

\begin{itemize}
\tightlist
\item
  \texttt{as.data.frame()} 함수는 벡터, 행렬 등을 인자로 받음.
\end{itemize}

\begin{Shaded}
\begin{Highlighting}[]
\NormalTok{윤하}
\end{Highlighting}
\end{Shaded}

\begin{verbatim}
##      Yunha
## 1 Password
## 2        4
## 3        8
## 4        6
\end{verbatim}

\begin{Shaded}
\begin{Highlighting}[]
\FunctionTok{class}\NormalTok{(윤하)}
\end{Highlighting}
\end{Shaded}

\begin{verbatim}
## [1] "data.frame"
\end{verbatim}

\begin{itemize}
\item
  데이터프레임은 row = observation, column = variable 개념이라, 열
  이름을 웬만하면 채우려고 하는데, 여기선 character vector 이름인
  Yunha를 차용한 모습.
\item
  벡터 이름 : 벡터 내용 = 변수 이름 : 관측치 느낌으로 해석한 듯.
\end{itemize}

\begin{Shaded}
\begin{Highlighting}[]
\NormalTok{matthew}
\end{Highlighting}
\end{Shaded}

\begin{verbatim}
##    A B C  D
## 가 1 4 7 10
## 나 2 5 8 11
## 다 3 6 9 12
\end{verbatim}

\begin{Shaded}
\begin{Highlighting}[]
\FunctionTok{class}\NormalTok{(matthew)}
\end{Highlighting}
\end{Shaded}

\begin{verbatim}
## [1] "data.frame"
\end{verbatim}

\hypertarget{uxc778uxb371uxc2f1}{%
\subsubsection{7) 인덱싱}\label{uxc778uxb371uxc2f1}}

\begin{Shaded}
\begin{Highlighting}[]
\NormalTok{Yunha[}\DecValTok{2}\NormalTok{] }\CommentTok{\# Yunha 벡터의 두 번째 요소}
\end{Highlighting}
\end{Shaded}

\begin{verbatim}
## [1] "4"
\end{verbatim}

\begin{Shaded}
\begin{Highlighting}[]
\NormalTok{mat[, }\DecValTok{1}\NormalTok{] }\CommentTok{\# mat 행렬의 첫 열}
\end{Highlighting}
\end{Shaded}

\begin{verbatim}
## 가 나 다 
##  1  2  3
\end{verbatim}

\begin{Shaded}
\begin{Highlighting}[]
\NormalTok{mat[}\DecValTok{2}\NormalTok{, }\DecValTok{2}\NormalTok{] }\CommentTok{\# mat 행렬의 (2, 2) 요소}
\end{Highlighting}
\end{Shaded}

\begin{verbatim}
## [1] 5
\end{verbatim}

\begin{Shaded}
\begin{Highlighting}[]
\NormalTok{matthew}\SpecialCharTok{$}\NormalTok{A }\CommentTok{\# matthew 데이터프레임의 변수 A에 속한 값}
\end{Highlighting}
\end{Shaded}

\begin{verbatim}
## [1] 1 2 3
\end{verbatim}

\begin{Shaded}
\begin{Highlighting}[]
\NormalTok{matthew[}\DecValTok{2}\NormalTok{, }\DecValTok{2}\NormalTok{] }\CommentTok{\# matthew 데이터프레임의 (2, 2) 요소}
\end{Highlighting}
\end{Shaded}

\begin{verbatim}
## [1] 5
\end{verbatim}

\begin{itemize}
\item
  데이터프레임의 인덱싱이 조금 더 깔끔한 느낌.
\item
  실제 작업에도 용이한바, R의 데이터분석은 대개 데이터프레임 형태를
  사용.
\item
  이를 \texttt{Pyhton}에서 구현하기 위해 \texttt{Pandas} 라이브러리를
  만듦.
\end{itemize}

\hypertarget{others}{%
\subsection{4. Others}\label{others}}

\textbf{\#} 주석처리

\begin{itemize}
\item
  코드 설명 작성 : \texttt{print(a)\ \#\ a를\ 출력한다}
\item
  해당 부분을 실행에서 제외 : \texttt{\#print(a)}
\end{itemize}

\textbf{업데이트} R, RStudio 모두 꾸준히 새 버전이 나오지만, 구 버전을
사용해도 큰 문제 없음

\begin{itemize}
\item
  R은 새로 설치하고 구 버전을 삭제하는 게 제일 간편. RStudio는 알아서
  최신의 R을 인식해 사용.
\item
  RStudio는 Help \(\to\) Check for Updates 기능을 활용.
\end{itemize}

\hypertarget{quiz}{%
\subsection{5. Quiz}\label{quiz}}

\textbf{초급} 다음의 행렬을 만들어보자

\begin{verbatim}
##         var1 var2 var3
## Case #1   12   21   32
## Case #2   17   22   34
## Case #3   19   25   35
\end{verbatim}

\textbf{중급} \texttt{datasets::iris} 데이터를 가져와 다음을 해결해보자

\begin{enumerate}
\def\labelenumi{\arabic{enumi})}
\item
  \texttt{iris} 변수의 이름을 \texttt{names()} 함수로 확인하라.
\item
  \texttt{iris} 관찰값, 변수의 개수를 \texttt{dim()}, \texttt{nrow()},
  \texttt{length()} 함수로 확인하라.
\item
  \texttt{iris} 처음 세 줄과 마지막 세 줄을 \texttt{head()},
  \texttt{tail()} 함수로 출력하라.
\end{enumerate}

\textbf{고급} 다음 코드의 문제점을 지적해보자. 수정본을 참고해도 좋다.

\begin{Shaded}
\begin{Highlighting}[]
\NormalTok{I}\SpecialCharTok{{-}}\NormalTok{DLE\_MEMBERS }\OtherTok{\textless{}{-}} \FunctionTok{c}\NormalTok{(}\StringTok{\textquotesingle{}소연\textquotesingle{}}\NormalTok{, }\StringTok{\textquotesingle{}미연\textquotesingle{}}\NormalTok{, }\StringTok{\textquotesingle{}민니\textquotesingle{}}\NormalTok{, }\StringTok{\textquotesingle{}우기\textquotesingle{}}\NormalTok{, }\StringTok{\textquotesingle{}슈화\textquotesingle{}}\NormalTok{)}

\NormalTok{I}\SpecialCharTok{{-}}\NormalTok{DLE\_LYLICS }\OtherTok{\textless{}{-}} \FunctionTok{c}\NormalTok{(}\StringTok{\textquotesingle{}Look at you 넌 못 감당해 날\textquotesingle{}}\NormalTok{, }
                  \StringTok{\textquotesingle{}I got to drink up now 네가 싫다 해도 좋아\textquotesingle{}}\NormalTok{, }
                  \StringTok{\textquotesingle{}Why are you cranky, boy? 뭘 그리 찡그려 너\textquotesingle{}}\NormalTok{, }
                  \StringTok{\textquotesingle{}미친 연이라 말해 What\textquotesingle{}}\NormalTok{s the loss to me ya}\StringTok{\textquotesingle{}, }
\StringTok{                  \textquotesingle{}}\NormalTok{사랑 그깟 거 따위 내 몸에 상처 하나도 어림없지}\StringTok{\textquotesingle{}, }
\StringTok{                  \textquotesingle{}}\NormalTok{Ye I}\StringTok{\textquotesingle{}m a Tomboy (Umm ah umm)\textquotesingle{}}\NormalTok{, }
                  \StringTok{\textquotesingle{}Ye I\textquotesingle{}}\NormalTok{ll be the }\FunctionTok{Tomboy}\NormalTok{ (Umm ah)}\StringTok{\textquotesingle{})}
\StringTok{                  }
\StringTok{TOMBOY \textless{}{-} data.frame(I{-}DLE\_MEMBERS, I{-}DLE\_LYLICS)}
\end{Highlighting}
\end{Shaded}

\begin{verbatim}
## Error: <text>:6:42: 예상하지 못한 기호(symbol)입니다.
## 5:                   'Why are you cranky, boy? 뭘 그리 찡그려 너', 
## 6:                   '미친 연이라 말해 What's
##                                             ^
\end{verbatim}

\begin{Shaded}
\begin{Highlighting}[]
\NormalTok{IDLE\_MEMBERS }\OtherTok{\textless{}{-}} \FunctionTok{c}\NormalTok{(}\StringTok{\textquotesingle{}소연\textquotesingle{}}\NormalTok{, }\StringTok{\textquotesingle{}미연\textquotesingle{}}\NormalTok{, }\StringTok{\textquotesingle{}민니\textquotesingle{}}\NormalTok{, }\StringTok{\textquotesingle{}우기\textquotesingle{}}\NormalTok{, }\StringTok{\textquotesingle{}슈화\textquotesingle{}}\NormalTok{)}

\NormalTok{IDLE\_LYLICS }\OtherTok{\textless{}{-}} \FunctionTok{c}\NormalTok{(}\StringTok{\textquotesingle{}Look at you 넌 못 감당해 날\textquotesingle{}}\NormalTok{, }
                 \StringTok{\textquotesingle{}I got to drink up now 네가 싫다 해도 좋아\textquotesingle{}}\NormalTok{, }
                 \StringTok{\textquotesingle{}Why are you cranky, boy? 뭘 그리 찡그려 너\textquotesingle{}}\NormalTok{, }
                 \StringTok{"미친 연이라 말해 What\textquotesingle{}s the loss to me ya"}\NormalTok{, }
                 \StringTok{\textquotesingle{}사랑 그깟 거 따위 내 몸에 상처 하나도 어림없지\textquotesingle{}}\NormalTok{, }
                 \StringTok{"Ye I\textquotesingle{}m a Tomboy (Umm ah umm)"}\NormalTok{, }
                 \StringTok{"Ye I\textquotesingle{}ll be the Tomboy (Umm ah)"}\NormalTok{)}
                  
\NormalTok{TOMBOY }\OtherTok{\textless{}{-}} \FunctionTok{data.frame}\NormalTok{(}\AttributeTok{MEMBERS =}\NormalTok{ IDLE\_MEMBERS[}\FunctionTok{c}\NormalTok{(}\DecValTok{3}\NormalTok{, }\DecValTok{4}\NormalTok{, }\DecValTok{1}\NormalTok{, }\DecValTok{2}\NormalTok{, }\DecValTok{4}\NormalTok{, }\DecValTok{3}\NormalTok{, }\DecValTok{5}\NormalTok{)], }
                     \AttributeTok{LYLICS =}\NormalTok{ IDLE\_LYLICS)}

\NormalTok{TOMBOY}
\end{Highlighting}
\end{Shaded}

\begin{verbatim}
##   MEMBERS                                         LYLICS
## 1    민니                    Look at you 넌 못 감당해 날
## 2    우기      I got to drink up now 네가 싫다 해도 좋아
## 3    소연     Why are you cranky, boy? 뭘 그리 찡그려 너
## 4    미연      미친 연이라 말해 What's the loss to me ya
## 5    우기 사랑 그깟 거 따위 내 몸에 상처 하나도 어림없지
## 6    민니                   Ye I'm a Tomboy (Umm ah umm)
## 7    슈화                 Ye I'll be the Tomboy (Umm ah)
\end{verbatim}

\hypertarget{day-2}{%
\section{Day 2}\label{day-2}}

\hypertarget{what-is-tidyverse}{%
\section{1. What is tidyverse}\label{what-is-tidyverse}}

\end{document}
